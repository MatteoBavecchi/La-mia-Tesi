\chapter{Ringraziamenti}\label{ch:ringraziamenti}
Sono molto fortunato.\\\\
 Ci sono diversi fattori che mi portano a questa conclusione, della maggior parte di questi
ne sono venuto a conoscenza durante i miei anni universitari, nei quali ritengo di aver avuto diversi momenti di crescita
e di riflessione.
Sono fortunato perchè ho avuto buoni esempi che mi hanno accompagnato durante la mia infanzia.
\\\\Mia mamma, sempre pronta ad assecondarmi e ad aiutarmi. E' sempre stata accanto a me, mi ha insegnato a voler bene e ad immedesimarmi nel prossimo prima di agire.
Mamma, a casa nostra gli abbracci non sono proprio così diffusi, anche se tu ne vorresti, te li darò più spesso.
\\\\Mio padre, una figura enorme, mente scientifica e pianificatrice, con dei valori scolpiti sulla pietra. Ho avuto difficoltà molte volte a decifrare i suoi messaggi,
ma ultimamente ci sto capendo un po' di più. Sono sicuro che molti stimoli che ho avuto in tenera età non erano casuali, ma mosse ben architettate
che seguivano un piano più ampio con l'obiettivo di farmi splendere. Il computer che mi mise in camera durante le elementari, la legge di Ohm scritta su un foglietto per farmi capire 
come potevo velocizzare la macchinina, i numerosi probemi che mi lasciava risolvere da solo, non per pigrizia ma perchè sapeva che dovevo imparare ad affrontarli.
 L'assenza completa dei \emph{no}, perchè fin da piccolo, davanti a qualsiasi mia richiesta, lui esponeva il suo parere, seguito da un \emph{vedi te}.
Le numerose responsabilità che avevo. La sua estrema arte dell'arrangiarsi, che mi mostrava per farmela interiorizzare il più possibile.
\\Babbo, non guardare che ultimamente litighiamo un po', ho una grande considerazione di te e di quello che fai.
\\\\Mia sorella, forte e determinata, essendo più grande di me ha avuto sempre un atteggiamento materno ed educativo nei miei confronti. 
Ultimamente l'ho riscoperta come una grande amica.
Vale, grazie per i tuoi consigli, ti voglio tanto bene.
\\\\
I miei nonni Grazia e Antonio, che mi hanno insegnato molte cose da piccolo,
vedono il mondo soprattutto soltanto con gli occhi di chi glielo racconta, e sempre più raramente con i loro. Io cerco di tenerli informati sulla mia vita e di spiegare
con il loro linguaggio le cose di tutti i giorni. Questo è il mio modo per ringraziarli.
\\Degli altri miei due nonni paterni, che non ci sono più, rimangono i ricordi, che rielaboro dopo anni e da cui traggo ancora degli insegnamenti.
Ricordo ad esempio mio nonno Elio, che da piccolo mi spiegò un giorno, portandomi in giro per il paese, l'importanza di sorridere quando qualcuno mi si rivolgeva,
 \emph{Quando qualcuno ti chiede come ti chiami, non dirlo con il broncio, sorridi!} mi disse.
 Questa frase, forse per lui spontanea e anche di poco conto, mi è tornata in mente qualche anno fa, e mi sono accorto quanto fosse stata importante.
 \\
 Mia nonna Attilia invece, era quella che in famiglia pronunciava più frequentemente la parola amore. In alcuni momenti mi sembrava ripetitiva, 
 adesso invece ne sento una profonda mancanza.
\\\\
Sono fortunato perchè ho dei bellissimi amici. Con ognuno di loro ho un interesse in comune, degli argomenti che non ci fanno mai stare in silenzio.
Qualcuno lo vedo spesso, qualcun'altro invece di rado. Ma non importa il tempo, comunque loro ci sono per me, e io per loro.
\\William, Marco, Cosimo, Daniele, Giada: con voi ho condiviso molti momenti importanti, e da voi ho sempre avuto consigli preziosi.
\\\\
Sono fortunato perchè durante il mio percorso universitario ho trovato persone che mi hanno aiutato tanto.
\\Niccolò e Andrea, siete stati determinanti per questo viaggio.
\\\\
 E' proprio vero che la saggezza può essere trasferita con insegnamenti, ma si conquista soprattutto con l'esperienza.
 Quando qualcuno vuole trasmetterci qualcosa lo fa parlandoci, 
 le parole arrivano a noi e il nostro cervello le elabora, ma questo deve essere pronto a recepire il messaggio racchiuso in quelle parole.
 Mi è successo proprio durante le superiori, il mio professore di ginnastica tutte le mattine, prima di andare in palestra, ci diceva sempre una frase: 
 \emph{siete voi i protagonisti della vostra vita!}. Questa frase l'ha ripetuta per cinque anni di fila. 
 Non avevo mai realmente capito perchè lui ci tenesse cosi tanto a dircela in continuazione.
 A distanza di circa 2 anni dalla mia Maturità, chissà in quale luogo e momento, mi torna in mente questa frase, e finalmente ne capisco il senso. Bellissimo.
 \\Ne ho fatto il mio motto.
