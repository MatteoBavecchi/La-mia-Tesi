\chapter{Collegare i microfrontends}\label{ch:composizione}
Per far convivere i progetti realizzati dai vari team esistono varie metodologie. Questo
 presuppone che i team, nella loro autonomia, debbano comunque scambiarsi un minimo di informazioni e di vincoli
che servono alla corretta integrazione delle app web, chiamiamo questo insieme di dati \emph{contratto tra team}.

\section*{Collegamento tra pagine con links}
La soluzione più semplice è quella del collegamento tra pagine con link URL: in questo modo il contratto consiste negli indirizzi delle pagine,
che i team dovranno scambiarsi tra di loro.
Questa soluzione è quella che rende i team il più autonomi e disaccoppiati possibile, inoltre
si ha grande robustezza, in quanto se un progetto si corrompe, questo non influenza minimamente gli altri,
che possono anche essere detenuti in server diversi.

Lo svantaggio è che in una pagina web è possibile contenere informazioni provenienti da un solo team.

Questa soluzione viene usata quando è richiesta una forte robustezza, oppure quando si deve implementare 
i microfrontends in una app legacy già esistente.


\section*{Composizione tramite iframes}
L' iframe è un elemento HTML che permette di incorporare un'altra pagina HTML all'interno di quella corrente. \cite{mozillaiframe}

Questo elemento può essere usato per comporre fragments provenienti da teams differenti, visualizzandoli su un'unica
pagina web.
Un team si impegnerà a realizzare la pagina ospitante e dovrà far presente agli altri attori lo spazio riservato
ai loro fragments per evitare problemi di visualizzazione, questo dato, oltre che agli indirizzi
URL delle varie pagine, farà parte del contratto tra i team.

Un grande svantaggio dell’uso degli iframes è la loro incompatibilità con i motori di ricerca.
Infatti le informazioni che vediamo nella pagina non sono in unico file HTML, di conseguenza il motore di ricerca non profila le informazioni 
contenute negli iframes.

\subsection*{Composizione con Ajax}
Un modo per superare il problema dei motori di ricerca ai quali sono affetti gli iframes è quello di 
caricare i file HTML con Ajax.

La pagina ospitante, detta \emph{wrapper} ha il compito di caricare nel proprio DOM ( Document Object Model) i fragments dei vari 
team, con l'ausilio di un codice javascript, che utilizza la funzione fetch().

Il problema principale però è che l’Ajax request è asincrona, 
questo porta a dei ritardi nel caricamento completo della pagina.

\subsection*{Utilizzo del Routing Server-side}
L’elemento che stiamo introducendo viene chiamato frontend proxy, cioè un web server che effettua routing,
 che intercetta le richieste con un certo percorso e le instrada al giusto fragment. In questo modo anche se
  i progetti dei vari team risiedono in spazi diversi, l’utente vedrà un URL omogeneo e non si accorgerà delle origini dei fragment.
Il funzionamento consiste in questi passaggi:
\begin{itemize}
    \item L’utente apre l’ URL “/foo/bar”. La richiesta raggiunge il frontend proxy
    \item Il frontend proxy confronta il path “/foo/bar/“8 con la propria routing table, con corrispondenza in prossimità della regola i
    \item  Il frontend proxy passa la richiesta al fragment associato
    \item Il fragment genera una risposta e la ritorna al frontdend proxy
    \item La risposta arriva infine all’utente
\end{itemize}





Il routing è un elemento centrale dell’architettura micro frontend, quindi è vitale investirci in qualità e testing.
Zalando, che applica l’architettura micro frontend al suo sito e-commerce, ha più di 800,000 Route definitions.

\pagebreak

\section*{Composizione Server-side}

\subsection*{Server Side Integration}

\subsubsection*{Modalità di caricamento dei fragments}

\subsection*{Alternative a SSI}
\subsubsection*{Zalando Tailor}
\subsubsection*{Podium}

\pagebreak
\section*{Composizione Client-side}

\subsection*{Comunicazione tra fragments}

\subsubsection*{Diretta}
\subsubsection*{Tramite il padre}
\subsubsection*{Event-bus}


\subsection*{Application Shell}


\subsubsection*{Routing a due livelli}




\pagebreak
\section*{Universal Rendering}