\chapter{Gestione delle risorse e stilizzazione}\label{ch:gestionerisorse}

Riuscire ad ottimizzare il caricamento delle risorse, come fogli di stile CSS, 
e mantenere un design coerente in tutti i sottoprogetti sono due
 grandi sfide della programmazione microfrontend, ne vediamo quindi delle strategie.


\section{Caricamento delle risorse}
Possono essere presenti nei progetti dei fragments dei file CSS e javascript, necessari
al giusto design e funzionamento logico del componente. Esistono diverse politiche 
atte a caricare le risorse necessarie a tutti i microfronend:

\subsection{Referencing con soluzioni client-rendered}
Il modo più semplice consiste nel caricare le risorse necessarie direttamente dalla pagina ospitante.
I team quindi dovranno fornire gli indirizzi dei file richiesti.
Per far risparmiare traffico agli utenti, di solito viene impostata una regola che fa mantenere ai client
le risorse in cache per un anno, in questo modo non devono essere scaricate tutte le volte.

Ma quando viene distribuito una nuova versione di file, i client hanno in cache delle versioni obsolete, e devono riscaricarle.
Per far si che avvenga, si appone al nome dei file il \emph{fingerprint}, un codice alfanumerico che deriva dal checksum del file.
Questa tecnica, chiamata \textbf{cache busting}, forza il browser dell'utente a riscaricare i file.
Per far si che i team possano distribuire autonomamente i file, senza comunicare ogni volta il nuovo nome del
fingerprint alla pagina ospitante, si include in questa dei nomi generici, che verranno poi reindirizzati dal webserver all'ultimo file aggiornato.



\subsection{Referencing con soluzioni server-rendered}
Se il progetto utilizza già il rendering lato server, è possibile includere le risorse nel codice
del fragment, in modo che quando questo viene sostituito dalla direttiva SSI dal webserver, va ad includere
alla pagina HTML anche i file necessari. Se però il fragment viene incluso nella pagina più volte, anche le risorse vengono 
scaricate ripetutamente. Inoltre i file javascript vengono eseguiti tante volte quanto vengono inclusi nella pagina,
il che va ad aumentare il carico sulla CPU del client e potrebbe creare errori di esecuzione.