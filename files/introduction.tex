\chapter{Introduzione}\label{ch:introduzione}
Negli ultimi due anni a causa della crisi pandemica abbiamo assistito ad un aumento esponenziale del commercio online:
secondo ISTAT nel nostro paese si è registrato un aumento del 50,2\% nelle vendite e-commerce.\cite{istat}
Il lavoro inoltre ha subito un cambiamento repentino, facendo migrare milioni di persone dall'ufficio alle loro mura domestiche.
Tutti questi fenomeni si sono realizzati grazie a internet, che ha accolto tutte quelle interazioni che fino a qualche mese prima
erano fatte fisicamente.
Questo ha portato ad avere sempre più traffico e una maggiore richiesta di robustezza, sicurezza e ridondanza da parte delle aziende.
Basti pensare al danno economico che hanno avuto le oltre \textbf{200 milioni} di attività commerciali che stavano pubblicizzando i loro beni su Facebook durante
le 14 ore di disservizi lo scorso 4 ottobre\cite{fbdown}.
Organizzare il lavoro per lo sviluppo di applicazioni web di grandi dimensioni, capaci di accogliere numeri sempre maggiori di utenti, non è per niente banale, ed è molto interessante
capire come suddividere responsabilità tra i vari team che contribuiscono alla sua realizzazione.

L'approccio più diffuso per affrontare il problema è quello di suddividere le persone per competenze (ovvero in modo \emph{orizzontale}), creando team che mettono in comune
figure con abilità dello stesso ambito. Il classico approccio monolitico infatti prevede due suddivisioni, la parte \emph{frontend}, che si occupa dell'esperienza 
utente e delle interfacce e la parte \emph{backend}, che invece gestisce server e basi di dati.
Ad esempio in un sito e-commerce possiamo trovare un team che si occupa della parte frontend, uno che cura 
i servizi di pagamento e uno che segue la parte backend. 

Quando il progetto aumenta di complessità, si sente la necessità 
di suddividere il lavoro in sotto-progetti, e l'approccio orizzontale potrebbe non essere la scelta migliore,
in quanto rallenta l'introduzione di nuove funzionalità.

Possiamo allora pensare di assegnare ad ogni team una parte del progetto, i quali dovranno portarlo al termine
interamente. Ogni team avrà bisogno quindi di competenze eterogenee al loro interno.

Abbiamo così il superamento dell'approccio monolitico, e possiamo parlare di microfrontend.

Questo già avviene analogamente nello sviluppo backend, dove l'applicazione viene vista come un
insieme di \emph{microservizi}.
