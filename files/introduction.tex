\chapter{Introduzione}\label{ch:introduzione}
Organizzare il lavoro per lo sviluppo di progetti web di grandi dimensioni non è per niente banale, e
può seguire diverse logiche di suddivisione di responsabilità tra le parti al quale contribuiscono.
L'approccio più diffuso è quello di suddividere le persone per competenze, creando team che mettono in comune
figure con abilità dello stesso ambito, che contribuiscono ad una parte del progetto complessivo.
Ad esempio in un sito e-commerce possiamo trovare un team che si occupa della parte frontend, uno che cura 
i servizi di pagamento e uno che segue la parte backend. Quando il progetto aumenta di complessità, si sente la necessità 
di suddividere il lavoro in sotto-progetti, e l'approccio orizzontale potrebbe non essere la scelta migliore,
in quanto rallenta l'introduzione di nuove funzionalità.

Possiamo allora pensare di assegnare ad ogni team una parte del progetto, i quali dovranno portarlo al termine
interamente. Ogni team avrà bisogno quindi di competenze eterogenee al loro interno.
