\chapter{Introduzione}\label{ch:introduzione}
Organizzare il lavoro per lo sviluppo di applicazioni web di grandi dimensioni non è per niente banale, ed è molto interessante
capire come suddividere responsabilità tra i vari team che contribuiscono alla sua realizzazione.

L'approccio più diffuso per affrontare il problema è quello di suddividere le persone per competenze (ovvero in modo \emph{orizzontale}), creando team che mettono in comune
figure con abilità dello stesso ambito. Il classico approccio monolitico infatti prevede due suddivisioni, la parte \emph{frontend}, che si occupa dell'esperienza 
utente e delle interfacce e la parte \emph{backend}, che invece gestisce server e basi di dati.
Ad esempio in un sito e-commerce possiamo trovare un team che si occupa della parte frontend, uno che cura 
i servizi di pagamento e uno che segue la parte backend. 

Quando il progetto aumenta di complessità, si sente la necessità 
di suddividere il lavoro in sotto-progetti, e l'approccio orizzontale potrebbe non essere la scelta migliore,
in quanto rallenta l'introduzione di nuove funzionalità.

Possiamo allora pensare di assegnare ad ogni team una parte del progetto, i quali dovranno portarlo al termine
interamente. Ogni team avrà bisogno quindi di competenze eterogenee al loro interno.

Abbiamo così il superamento dell'approccio monolitico, e possiamo parlare di microfrontend.

Questo già avviene analogamente nello sviluppo backend, dove l'applicazione viene vista come un
insieme di \emph{microservizi} disaccoppiati e con granularità fine.
