\chapter{Microfrontend}\label{ch:chapter1}
L'obiettivo della tecnologia microfrontend è quello di superare l'approccio monolitico, che vede
lo sviluppo di applicazioni web suddiviso in due teams: backend e frontend.

Si fa questo vedendo un'applicazione web come un insieme
di elementi, chiamati fragment o microfrontend, molto disaccoppiati tra di loro
e con la più bassa granuralità, ovvero con la funzionalità più minimale possibile.

Ogni microfrontend viene sviluppato da un team, che potrà lavorare con autonomia. 
Essendo i singoli microfrontend autonomi,
questi possono funzionare anche se estratti dalla applicazione web che li contiene,
 e il malfunzionamento di un singolo microfrontend
non compromette la stabilità degli altri.



\subsection*{Vantaggi}
\begin{itemize}
    \item \textbf{Ottimizzare sviluppo di funzionalità:}
Nell’approccio orizzontale quando si vuole sviluppare una nuova funzionalità ci si
 deve accordare tra vari team e fare molti incontri per discutere. Con i microfrontend,
  tutte le persone coinvolte nell’implementazione di una nuova funzionalità sono nello 
  stesso team, e quindi è tutto più veloce ed efficiente. 
    \item \textbf{Abbandono del frontend monolitico:}
La maggior parte delle acrchitetture oggi non hanno il concetto di scalablità nel 
frontend, ma per questo hanno un concetto monolitico.
Si può pensare di dividere il backend e il frontend, oppure di rendere il backend 
l’insieme di tanti microservizi, ma il frontend rimane unico.
Con microfrontend le applicazioni, incluse il frontend, si dividono in sistemi verticali
 più piccoli. Ogni team controlla la sua piccola parte di frontend. Questo porta diversi
  benefici, come l’isolamento dei rischi di fallimento, e maggiore predicibilità, in quanto
   si ha pezzi più piccoli di sistema che non condividono molti stati con il resto dell’applicazione.
    \item \textbf{Abilità nel cambiare:}
Strumenti di sviluppo e framework evolvono continuamente. E’ essenziale per un team 
essere in grado di adottare una nuova tecnologia quando questo ha senso. Ci sono esempi
 di aziende che hanno avuto bisogno di adottare strumenti diversi che hanno dovuto fare 
 delle transizioni epocali, alcune durate diversi anni. Come il caso di Github, che ha 
 impiegato molto tempo per eliminare le dipendenze da JQuery dal loro codice.
  Con l’approccio microfrontend questi cambiamenti sono più rapidi e possono essere
   fatti modularmente.


\item \textbf{Indipendenza:}
I frammenti di ogni team sono autonomi, ovvero non hanno dipendenze condivise a runtime.
 Questo permette ad ogni team di introdurre funzionalità senza consultare altri.

L’indipendenza però porta sicuramente a costi aggiuntivi. Si potrebbe pensare che sia
 più semplice quindi fare un unica applicazione di grandi dimensioni di cui ogni team
  è responsabile di una parte diversa. Il problema sta nel fatto che la comunicazione 
  tra i vari team è costosa e porta molti ritardi.
Quindi si preferisce addirittura la ridondanza a favore di più autonomia e velocità di
 implementazioni.

\end{itemize}

\subsection*{Svantaggi}
\begin{itemize}

\item \textbf{Ridondanza:}
In informatica si è addestrati a ridurre al minimo ridondanze, che si parli di normalizzazione
 dei dati nei database o di estrarre parti di codici condivisi per farne una funzione.
Anche nel caso dello sviluppo frontend ci sono degli episodi nei quali la ridondanza può
 essere molto costosa: come ad esempio quando viene trovato un bug in una libreria e 
 questo viene risolto da un team, questo poi dovrà essere comunicato agli altri e questi 
 dovranno provvedere a risolverlo autonomamente. Oppure quando si rende un processo più 
 veloce, anche in questo caso va comunicato agli altri team e questi dovranno apprendere
  la scoperta.
Si adotta quindi un approccio microfrontend quando i costi associati a ridondanze sono 
inferiori agli impatti negativi a uno sviluppo frontend monolitico, che porta a forti 
dipendenze tra team.




\item \textbf{Inconsistenza:}
I database usati nell’applicazione devono essere a disposizione di tutti i team. 
Quindi quello che si fa di solito è replicare il database per tutti i team. 
Tutte queste copie vengono aggiornate regolarmente per mantenere la consistenza e la coerenza.
 Ma ci potrebbero essere dei ritardi. Di solito si parla di millisecondi o al massimo secondi,
  in casi peggiori invece l’attesa potrebbe essere più lunga.
Questo è un altro compromesso che privilegia la robustezza alla coerenza garantita.

\item \textbf{Eterogeneità:}
Potrebbe essere controverso avere la libertà di utilizzare tecnologie diverse tra i vati team.
 Ovviamente se ogni team usa una diversa tecnologia lo scambio di pareri o di competenze 
 tra team diventa più difficile
Ma questo non deve essere un obbligo. Da un elenco di tecnologie collaudate un team si
 sentirà libero di scegliere quella più accettabile.
\end{itemize}