\chapter{Conclusioni}\label{ch:conclusioni}
Come lo sono i microservizi per il lato backend, i microfrontend sono un efficace soluzione per dividere i compiti tra team omogenei
e realizzare un progetto web modulare e reattivo ai cambiamenti. Ogni team produce delle applicazioni indipendenti, 
che possono essere estratte dal contesto e riproposte in altri progetti che richiedono funzionalità analoghe.
Numerose aziende hanno migrato verso questa tecnologia, una di queste è la tedesca Zalando, leader nel settore e-commerce, che con
il progetto Mosaic ha rilasciato pubblicamente una raccolta di servizi e librerie per aiutare proprietari di grandi siti web a migrare 
dall'approccio monolitico, contribuendo la diffusione di tale pratica rendendola più accessibile \cite{mosaic}, non si deve trascurare il fatto che 
realizzare un progetto web con questa architettura è molto più complesso di farlo con l'approccio monolitico, o almeno inizialmente.

Una parola ricorrente in questo elaborato è \textbf{composizione}: ovvero il far convivere nella solita finestra del browser
più fragments provenienti da team diversi. La composizione può essere effettuata prima di far arrivare la pagina al client, oppure
dopo, delegando l'operazione al browser.

Molto spesso si è parlato di \textbf{autonomia}, che non vuol dire isolamento: i team lavorano per lo più internamente, ma 
a seconda delle tecnologie utilizzate, devono presentare agli altri un \emph{contratto}, delle informazioni che devono essere condivise
per mettere in fase l'intero progetto. Queste informazioni però devono restare minime, 
perche quanto più il contratto è pieno di "clausole" da rispettare, tanto più il progetto si complica.

La piattaforma Leonardo X2030 è l'esempio perfetto per la messa in opera di un'architettura microfrontend:
contiene molti strumenti, ha un'interfaccia modulare personalizzabile al massimo, è aggioranta continuamente
per essere compatibile con le nuove tecnologie, come rilievi con droni o gestione di big data sfruttando l'intelligenza artificiale.
Tutte queste funzionalità eterogenee devono comunicare tra di loro, e devono essere racchiuse in un interfaccia che le
renda utilizzabili contemporaneamente al personale di diversi enti.
